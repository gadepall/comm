\documentclass[journal,12pt,onecolumn]{IEEEtran}
%
\usepackage[utf8]{inputenc}
\usepackage[table]{xcolor}
\usepackage{graphicx}
\usepackage{tfrupee}
%\usepackage{amssymb}
%\usepackage{relsize}
\usepackage[cmex10]{amsmath}
%\usepackage{amsthm}
\interdisplaylinepenalty=2500
%\savesymbol{iint}
%\usepackage{txfonts}
%\restoresymbol{TXF}{iint}
%\usepackage{wasysym}
%\usepackage[active,tightpage]{preview}
\usepackage{amsthm}
\usepackage{mathrsfs}
\usepackage{txfonts}
\usepackage{stfloats}
\usepackage{cite}
\usepackage{tikz}
\usepackage{cases}
\usepackage{subfig}
%\usepackage{xtab}
\usepackage{longtable}
\usepackage{multirow}
%\usepackage{algorithm}
%\usepackage{algpseudocode}
\usepackage{enumitem}
\usepackage{mathtools}
%\usepackage{iithtlc}
\usepackage{textcomp}
%\usetikzlibrary{shapes,arrows}
%\usetikzlibrary{positioning,calc}
%\usepackage{stmaryrd}

%     \usepackage[latin1]{inputenc}
       \usepackage{fullpage}
       \usepackage{color}
       \usepackage{array}
       \usepackage{longtable}
       \usepackage{calc}
       \usepackage{multirow}
       \usepackage{hhline}
       \usepackage{ifthen}

%\usepackage{wasysym}
%\newcounter{MYtempeqncnt}
\DeclareMathOperator*{\Res}{Res}
%\renewcommand{\baselinestretch}{2}
\renewcommand\thesection{\arabic{section}}
\renewcommand\thesubsection{\thesection.\arabic{subsection}}
\renewcommand\thesubsubsection{\thesubsection.\arabic{subsubsection}}

\renewcommand\thesectiondis{\arabic{section}}
\renewcommand\thesubsectiondis{\thesectiondis.\arabic{subsection}}
\renewcommand\thesubsubsectiondis{\thesubsectiondis.\arabic{subsubsection}}

% correct bad hyphenation here
\hyphenation{op-tical net-works semi-conduc-tor}


\begin{document}
%


\newtheorem{theorem}{Theorem}[section]
\newtheorem{problem}{Problem}
\newtheorem{proposition}{Proposition}[section]
\newtheorem{lemma}{Lemma}[section]
\newtheorem{corollary}[theorem]{Corollary}
\newtheorem{example}{Example}[section]
\newtheorem{definition}{Definition}[section]
%\newtheorem{algorithm}{Algorithm}[section]
%\newtheorem{cor}{Corollary}
\newcommand{\BEQA}{\begin{eqnarray}}
\newcommand{\EEQA}{\end{eqnarray}}
\newcommand{\define}{\stackrel{\triangle}{=}}

\bibliographystyle{IEEEtran}
%\bibliographystyle{ieeetr}


\providecommand{\mbf}{\mathbf}
\providecommand{\pr}[1]{\ensuremath{\Pr\left(#1\right)}}
\providecommand{\qfunc}[1]{\ensuremath{Q\left(#1\right)}}
\providecommand{\sbrak}[1]{\ensuremath{{}\left[#1\right]}}
\providecommand{\lsbrak}[1]{\ensuremath{{}\left[#1\right.}}
\providecommand{\rsbrak}[1]{\ensuremath{{}\left.#1\right]}}
\providecommand{\brak}[1]{\ensuremath{\left(#1\right)}}
\providecommand{\lbrak}[1]{\ensuremath{\left(#1\right.}}
\providecommand{\rbrak}[1]{\ensuremath{\left.#1\right)}}
\providecommand{\cbrak}[1]{\ensuremath{\left\{#1\right\}}}
\providecommand{\lcbrak}[1]{\ensuremath{\left\{#1\right.}}
\providecommand{\rcbrak}[1]{\ensuremath{\left.#1\right\}}}
\theoremstyle{remark}
\newtheorem{rem}{Remark}
\newcommand{\sgn}{\mathop{\mathrm{sgn}}}
\providecommand{\abs}[1]{\left\vert#1\right\vert}
\providecommand{\res}[1]{\Res\displaylimits_{#1}} 
\providecommand{\norm}[1]{\lVert#1\rVert}
\providecommand{\mtx}[1]{\mathbf{#1}}
\providecommand{\mean}[1]{E\left[ #1 \right]}
\providecommand{\fourier}{\overset{\mathcal{F}}{ \rightleftharpoons}}
%\providecommand{\hilbert}{\overset{\mathcal{H}}{ \rightleftharpoons}}
\providecommand{\system}{\overset{\mathcal{H}}{ \longleftrightarrow}}
	%\newcommand{\solution}[2]{\textbf{Solution:}{#1}}
\newcommand{\solution}{\noindent \textbf{Solution: }}
\providecommand{\dec}[2]{\ensuremath{\overset{#1}{\underset{#2}{\gtrless}}}}
\newcommand{\suma}{\Large$+$}
\newcommand{\inte}{$\displaystyle \int$}
\newcommand{\derv}{\huge$\frac{d}{dt}$}

%\numberwithin{equation}{section}
%\numberwithin{problem}{section}

\def\putbox#1#2#3{\makebox[0in][l]{\makebox[#1][l]{}\raisebox{\baselineskip}[0in][0in]{\raisebox{#2}[0in][0in]{#3}}}}
     \def\rightbox#1{\makebox[0in][r]{#1}}
     \def\centbox#1{\makebox[0in]{#1}}
     \def\topbox#1{\raisebox{-\baselineskip}[0in][0in]{#1}}
     \def\midbox#1{\raisebox{-0.5\baselineskip}[0in][0in]{#1}}

        \def\inputGnumericTable{}                                 %%

%\title{
%	\logo{Matrix Analysis through Octave}
%}
\title{Simulation and Performance Evaluation of Equalizer and Synchronization Algorithms for  Narrow band Modem}
%}


% paper title
% can use linebreaks \\ within to get better formatting as desired
%\title{Matrix Analysis through Octave}
%
%
% author names and IEEE memberships
% note positions of commas and nonbreaking spaces ( ~ ) LaTeX will not break
% a structure at a ~ so this keeps an author's name from being broken across
% two lines.
% use \thanks{} to gain access to the first footnote area
% a separate \thanks must be used for each paragraph as LaTeX2e's \thanks
% was not built to handle multiple paragraphs
%

%\author{G V V Sharma$^{*}$% <-this % stops a space
%\thanks{*The author is with the Department
%of Electrical Engineering, Indian Institute of Technology, Hyderabad
%502285 India e-mail:  gadepall@iith.ac.in.}% <-this % stops a space
%%\thanks{J. Doe and J. Doe are with Anonymous University.}% <-this % stops a space
%%\thanks{Manuscript received April 19, 2005; revised January 11, 2007.}}
%}
% note the % following the last \IEEEmembership and also \thanks - 
% these prevent an unwanted space from occurring between the last author name
% and the end of the author line. i.e., if you had this:
% 
% \author{....lastname \thanks{...} \thanks{...} }
%                     ^------------^------------^----Do not want these spaces!
%
% a space would be appended to the last name and could cause every name on that
% line to be shifted left slightly. This is one of those "LaTeX things". For
% instance, "\textbf{A} \textbf{B}" will typeset as "A B" not "AB". To get
% "AB" then you have to do: "\textbf{A}\textbf{B}"
% \thanks is no different in this regard, so shield the last } of each \thanks
% that ends a line with a % and do not let a space in before the next \thanks.
% Spaces after \IEEEmembership other than the last one are OK (and needed) as
% you are supposed to have spaces between the names. For what it is worth,
% this is a minor point as most people would not even notice if the said evil
% space somehow managed to creep in.



% The paper headers
%\markboth{Journal of \LaTeX\ Class Files,~Vol.~6, No.~1, January~2007}%
%{Shell \MakeLowercase{\textit{et al.}}: Bare Demo of IEEEtran.cls for Journals}
% The only time the second header will appear is for the odd numbered pages
% after the title page when using the twoside option.
% 
% *** Note that you probably will NOT want to include the author's ***
% *** name in the headers of peer review papers.                   ***
% You can use \ifCLASSOPTIONpeerreview for conditional compilation here if
% you desire.




% If you want to put a publisher's ID mark on the page you can do it like
% this:
%\IEEEpubid{0000--0000/00\$00.00~\copyright~2007 IEEE}
% Remember, if you use this you must call \IEEEpubidadjcol in the second
% column for its text to clear the IEEEpubid mark.



% make the title area
\maketitle

%\tableofcontents


%\begin{abstract}
%Development of Encoder and Decoder Modules for Low-Density
%Parity Check Code for DVB-S2 
%%%\boldmath
%In this paper, a new hands-on technique of teaching digital system design through single board computers is introduced. Besides digital design, this approach also provides a quick introduction to C and assembly language programming.  The problem of designing a decade counter is divided into a series of simple problems that are used to introduce boolean logic, combinational logic, Karnaugh maps and algorithmic state machines  in sequence.  The single board computers used for instruction are Ardunio and Raspberry Pi.   
%%
%\end{abstract}
%\section{Proposal Summary}
%
%\bigskip
%
%\begin{tabular}{@{}lcl}
%{\bf A. Project title} &:& 
%%\section{Methodology}
%Simulation and performance evaluation of Equalizer\\
% 
%%Development of a Robust Equalizer with time/frequency \\
%%\\[1ex]
%& & and Synchronization algorithms for  Narrow band Modem
%%synchronization at the receiver for a V/UHF Fading Channel
%\\[1ex]
%%& &Forward Error Correction (FEC)
%%\\%[1ex]
%%               & &
%%\\ [1ex]              
%%   & &            and Blocklengths for Digital Video Broadcast (DVB-S2)\\[1ex]
%{\bf B. Nodal Institution} &:& Indian Institute of Technology Hyderabad \\[1ex]
%{\bf C. Funding Agency} &:& Bharat Electronics Limited \\[1ex]
%{\bf D. Principal Investigator} &:& Dr. GVV Sharma\\
%                                & & Department of Electrical Engineering \\
%                                & & Indian Institute of Technology Hyderabad \\
%                                & & Kandi, Sangareddy 502\,285 \\[1ex]
%%                                & & Email: lakshminatarajan@iith.ac.in\\[1ex]
%%{\bf \,\,\,Co-Principal Investigator 1} &:& Dr. G V V Sharma \\
%%                                & & Department of Electrical Engineering \\
%%                                & & Indian Institute of Technology Hyderabad \\
%%                                & & Kandi, Sangareddy 502\,285 \\
%%                                & & Email: gadepall@iith.ac.in\\[1ex]
%%{\bf \,\,\,Co-Principal Investigator 2} &:& Dr. Abhinav Kumar \\
%%                                & & Department of Electrical Engineering \\
%%                                & & Indian Institute of Technology Hyderabad \\
%%                                & & Kandi, Sangareddy 502\,285 \\
%%                                & & Email: abhinavkumar@iith.ac.in\\[1ex]
%{\bf E. Total Estimated Cost} &:& \rupee.~8,07,120/- (inclusive of GST@18\%) \\[1ex]
%{\bf F. Duration} &:& 3 months \\[1ex]
%{\bf G. Details of Budget} &:& \hspace*{-3cm}
%\input{./tables/summary.tex}
%\\\\[1ex]
%%{\bf H. Advance Payable } &:& 
%%\input{./tables/milestones.tex}
%%\rupee. ~1,84,080 (25\% of Budget )
%%\\[1ex]
%\end{tabular}

%\documentclass[11pt]{article}

%\usepackage{array}
%\usepackage{tabularx}
%\usepackage{amssymb}
%\usepackage{mathtools}
%\usepackage{amsmath}
%\usepackage{graphicx}
%
%\begin{document}
%\title{\textbf{Analysis of LDPC codes in DVB-S2}}
%\maketitle
\section{Specifications}
To develop a robust equalizer with time/frequency synchronization at the receiver to  sustain fading effects of
V/UHF channel as per the  specifications in Table \ref{table:specs}
\begin{table}
\centering
\subsection{Specifications}
To develop a robust equalizer with time/frequency synchronization at the receiver to  sustain fading effects of
V/UHF channel as per the  specifications in Table \ref{table:specs}
\begin{table}
\centering
\subsection{Specifications}
To develop a robust equalizer with time/frequency synchronization at the receiver to  sustain fading effects of
V/UHF channel as per the  specifications in Table \ref{table:specs}
\begin{table}
\centering
\subsection{Specifications}
To develop a robust equalizer with time/frequency synchronization at the receiver to  sustain fading effects of
V/UHF channel as per the  specifications in Table \ref{table:specs}
\begin{table}
\centering
\input{./tables/specs.tex}
\caption{}
\label{table:specs}
\end{table}
\subsection{Sequence of Steps}
\begin{enumerate}

\item Simulation of suitable Preamble detection algorithm.
\item Simulation of suitable channel estimation and equalization algorithms for mitigating Rayleigh fading channel effects.
\item  Simulation of suitable timing and frequency algorithms for narrow band waveform (coherent TCM-8PSK). Convergence of timing, frequency and equalization algorithms in limited preamble symbols provided (19.5 bytes which is equal to 156 bits or 78 symbols for  TCM 8PSK modulation) and efficient utilization of hardware resources (required all these algorithms shall take max 40\% of FPGA resources).
\item Performance evaluation and Simulation of equalizer and synchronization algorithms should to be done in fixed point.
\item MATLAB codes should be hardware implementable and consider FPGA resources restrictions. Matrix inversions in MATLAB code should not be there.

\end{enumerate}
\subsection{Technology Overview}
Preamble detection can be done using a special kind of correlation as in \cite{frame_offset}.  A symbol detection technique in the presence of time offsets is available in \cite{time_offset}.  A correlation based technique for frequency offset estimation is provided in \cite{freq_offset}. A DFSE based equalization technique based on the Viterbi algorithm is provided in \cite{dfse_viterbi}. A constant modulus decision directed algorithm is proposed in \cite{cmdd}. A channel impulse response based sparse equalization and synchronization technique is given in \cite{cir_sparse}.  Blind equalization techniques are listed in \cite{blind}.

All the above techniques will be analyzed before before finalizing the algorithms for preamble detection, synchronization and equalization.
%\renewcommand{\theequation}{\theenumi}
%\begin{enumerate}[label=\arabic*.,ref=\thesubsection.\theenumi]
%\numberwithin{equation}{enumi}
%	
%\item
%\label{ch1_lp1}
%	Graphically obtain a solution to the following 
%	\begin{align}
%\max_{\mbf{x}}	6x_1 + 5x_2
%	\end{align}
%	with constraints
%	\begin{align}
%	x_1 + x_2 &\leq 5\\
%	3x_1 + 2x_2 &\leq 12\\
%	\text{ where } x_1,x_2 &\geq 0
%	\end{align}
%
%%
%\solution
%The following program plots the solution in Fig. \ref{fig.4.1}
%%	
%\begin{lstlisting}
%codes/optimization/4.1.py
%\end{lstlisting}
%
%%
%\begin{figure}[!ht]
%\centering
%\includegraphics[width=\columnwidth]{./optimization/figs/4.1.eps}
%\caption{ The cost function intersects with the two constraints at $\mbf{x} = \brak{2,3}$. }
%\label{fig.4.1}	
%\end{figure}
%%
%\item
%	Now use {\em cvxpy} to obtain a solution to problem \ref{ch1_lp1}.
%
%\solution
%The given problem is expressed as follows
%%
%\begin{align}
%\min_{\mbf{x}}	\mbf{c}^{T}\mbf{x}\quad s.t.
%\\
%\mbf{A}\mbf{x} \preceq \mbf{b}
%\end{align}
%%
%where
%%
%\begin{equation}
%\mbf{c}
%=
%\begin{pmatrix}
%-6
%\\
%-5
%\end{pmatrix},
%\mbf{A} = 
%\begin{pmatrix}
%1 & 1
%\\
%3 & 2
%\\
%-1 & 0
%\\
%0 & -1
%\end{pmatrix},
%\mbf{b}
%= 
%\begin{pmatrix}
%5
%\\
%12
%\\
%0
%\\
%0 
%\end{pmatrix}
%\end{equation}
%%	
%The desired solution is then obtained using the following program.
%%\begin{lstlisting}
%%codes/optimization/4.2.py
%%\end{lstlisting}
%%
%%
%%\item
%%Repeat the previous exercise using {\em cvxpy}
%%
%%\solution
%\begin{lstlisting}
%codes/optimization/4.2-cvx.py
%\end{lstlisting}
%
%\item
%	Verify your solution to the above problem using the method of Lagrange multipliers.
%
%%
%\item
%	 Maximise $5x_1 + 3x_2$ w.r.t the constraints
%	 \begin{align}
%	 x_1 + x_2 &\leq 2 \nonumber\\
%	 5x_1 + 2x_2 &\leq 10 \nonumber\\
%	 3x_1 + 8x_2 &\leq 12 \nonumber\\
%	 \text{ where } x_1,x_2 &\geq 0 \nonumber
%	 \end{align}	
%
%\end{enumerate}
%%
%

\caption{}
\label{table:specs}
\end{table}
\subsection{Sequence of Steps}
\begin{enumerate}

\item Simulation of suitable Preamble detection algorithm.
\item Simulation of suitable channel estimation and equalization algorithms for mitigating Rayleigh fading channel effects.
\item  Simulation of suitable timing and frequency algorithms for narrow band waveform (coherent TCM-8PSK). Convergence of timing, frequency and equalization algorithms in limited preamble symbols provided (19.5 bytes which is equal to 156 bits or 78 symbols for  TCM 8PSK modulation) and efficient utilization of hardware resources (required all these algorithms shall take max 40\% of FPGA resources).
\item Performance evaluation and Simulation of equalizer and synchronization algorithms should to be done in fixed point.
\item MATLAB codes should be hardware implementable and consider FPGA resources restrictions. Matrix inversions in MATLAB code should not be there.

\end{enumerate}
\subsection{Technology Overview}
Preamble detection can be done using a special kind of correlation as in \cite{frame_offset}.  A symbol detection technique in the presence of time offsets is available in \cite{time_offset}.  A correlation based technique for frequency offset estimation is provided in \cite{freq_offset}. A DFSE based equalization technique based on the Viterbi algorithm is provided in \cite{dfse_viterbi}. A constant modulus decision directed algorithm is proposed in \cite{cmdd}. A channel impulse response based sparse equalization and synchronization technique is given in \cite{cir_sparse}.  Blind equalization techniques are listed in \cite{blind}.

All the above techniques will be analyzed before before finalizing the algorithms for preamble detection, synchronization and equalization.
%\renewcommand{\theequation}{\theenumi}
%\begin{enumerate}[label=\arabic*.,ref=\thesubsection.\theenumi]
%\numberwithin{equation}{enumi}
%	
%\item
%\label{ch1_lp1}
%	Graphically obtain a solution to the following 
%	\begin{align}
%\max_{\mbf{x}}	6x_1 + 5x_2
%	\end{align}
%	with constraints
%	\begin{align}
%	x_1 + x_2 &\leq 5\\
%	3x_1 + 2x_2 &\leq 12\\
%	\text{ where } x_1,x_2 &\geq 0
%	\end{align}
%
%%
%\solution
%The following program plots the solution in Fig. \ref{fig.4.1}
%%	
%\begin{lstlisting}
%codes/optimization/4.1.py
%\end{lstlisting}
%
%%
%\begin{figure}[!ht]
%\centering
%\includegraphics[width=\columnwidth]{./optimization/figs/4.1.eps}
%\caption{ The cost function intersects with the two constraints at $\mbf{x} = \brak{2,3}$. }
%\label{fig.4.1}	
%\end{figure}
%%
%\item
%	Now use {\em cvxpy} to obtain a solution to problem \ref{ch1_lp1}.
%
%\solution
%The given problem is expressed as follows
%%
%\begin{align}
%\min_{\mbf{x}}	\mbf{c}^{T}\mbf{x}\quad s.t.
%\\
%\mbf{A}\mbf{x} \preceq \mbf{b}
%\end{align}
%%
%where
%%
%\begin{equation}
%\mbf{c}
%=
%\begin{pmatrix}
%-6
%\\
%-5
%\end{pmatrix},
%\mbf{A} = 
%\begin{pmatrix}
%1 & 1
%\\
%3 & 2
%\\
%-1 & 0
%\\
%0 & -1
%\end{pmatrix},
%\mbf{b}
%= 
%\begin{pmatrix}
%5
%\\
%12
%\\
%0
%\\
%0 
%\end{pmatrix}
%\end{equation}
%%	
%The desired solution is then obtained using the following program.
%%\begin{lstlisting}
%%codes/optimization/4.2.py
%%\end{lstlisting}
%%
%%
%%\item
%%Repeat the previous exercise using {\em cvxpy}
%%
%%\solution
%\begin{lstlisting}
%codes/optimization/4.2-cvx.py
%\end{lstlisting}
%
%\item
%	Verify your solution to the above problem using the method of Lagrange multipliers.
%
%%
%\item
%	 Maximise $5x_1 + 3x_2$ w.r.t the constraints
%	 \begin{align}
%	 x_1 + x_2 &\leq 2 \nonumber\\
%	 5x_1 + 2x_2 &\leq 10 \nonumber\\
%	 3x_1 + 8x_2 &\leq 12 \nonumber\\
%	 \text{ where } x_1,x_2 &\geq 0 \nonumber
%	 \end{align}	
%
%\end{enumerate}
%%
%

\caption{}
\label{table:specs}
\end{table}
\subsection{Sequence of Steps}
\begin{enumerate}

\item Simulation of suitable Preamble detection algorithm.
\item Simulation of suitable channel estimation and equalization algorithms for mitigating Rayleigh fading channel effects.
\item  Simulation of suitable timing and frequency algorithms for narrow band waveform (coherent TCM-8PSK). Convergence of timing, frequency and equalization algorithms in limited preamble symbols provided (19.5 bytes which is equal to 156 bits or 78 symbols for  TCM 8PSK modulation) and efficient utilization of hardware resources (required all these algorithms shall take max 40\% of FPGA resources).
\item Performance evaluation and Simulation of equalizer and synchronization algorithms should to be done in fixed point.
\item MATLAB codes should be hardware implementable and consider FPGA resources restrictions. Matrix inversions in MATLAB code should not be there.

\end{enumerate}
\subsection{Technology Overview}
Preamble detection can be done using a special kind of correlation as in \cite{frame_offset}.  A symbol detection technique in the presence of time offsets is available in \cite{time_offset}.  A correlation based technique for frequency offset estimation is provided in \cite{freq_offset}. A DFSE based equalization technique based on the Viterbi algorithm is provided in \cite{dfse_viterbi}. A constant modulus decision directed algorithm is proposed in \cite{cmdd}. A channel impulse response based sparse equalization and synchronization technique is given in \cite{cir_sparse}.  Blind equalization techniques are listed in \cite{blind}.

All the above techniques will be analyzed before before finalizing the algorithms for preamble detection, synchronization and equalization.
%\renewcommand{\theequation}{\theenumi}
%\begin{enumerate}[label=\arabic*.,ref=\thesubsection.\theenumi]
%\numberwithin{equation}{enumi}
%	
%\item
%\label{ch1_lp1}
%	Graphically obtain a solution to the following 
%	\begin{align}
%\max_{\mbf{x}}	6x_1 + 5x_2
%	\end{align}
%	with constraints
%	\begin{align}
%	x_1 + x_2 &\leq 5\\
%	3x_1 + 2x_2 &\leq 12\\
%	\text{ where } x_1,x_2 &\geq 0
%	\end{align}
%
%%
%\solution
%The following program plots the solution in Fig. \ref{fig.4.1}
%%	
%\begin{lstlisting}
%codes/optimization/4.1.py
%\end{lstlisting}
%
%%
%\begin{figure}[!ht]
%\centering
%\includegraphics[width=\columnwidth]{./optimization/figs/4.1.eps}
%\caption{ The cost function intersects with the two constraints at $\mbf{x} = \brak{2,3}$. }
%\label{fig.4.1}	
%\end{figure}
%%
%\item
%	Now use {\em cvxpy} to obtain a solution to problem \ref{ch1_lp1}.
%
%\solution
%The given problem is expressed as follows
%%
%\begin{align}
%\min_{\mbf{x}}	\mbf{c}^{T}\mbf{x}\quad s.t.
%\\
%\mbf{A}\mbf{x} \preceq \mbf{b}
%\end{align}
%%
%where
%%
%\begin{equation}
%\mbf{c}
%=
%\begin{pmatrix}
%-6
%\\
%-5
%\end{pmatrix},
%\mbf{A} = 
%\begin{pmatrix}
%1 & 1
%\\
%3 & 2
%\\
%-1 & 0
%\\
%0 & -1
%\end{pmatrix},
%\mbf{b}
%= 
%\begin{pmatrix}
%5
%\\
%12
%\\
%0
%\\
%0 
%\end{pmatrix}
%\end{equation}
%%	
%The desired solution is then obtained using the following program.
%%\begin{lstlisting}
%%codes/optimization/4.2.py
%%\end{lstlisting}
%%
%%
%%\item
%%Repeat the previous exercise using {\em cvxpy}
%%
%%\solution
%\begin{lstlisting}
%codes/optimization/4.2-cvx.py
%\end{lstlisting}
%
%\item
%	Verify your solution to the above problem using the method of Lagrange multipliers.
%
%%
%\item
%	 Maximise $5x_1 + 3x_2$ w.r.t the constraints
%	 \begin{align}
%	 x_1 + x_2 &\leq 2 \nonumber\\
%	 5x_1 + 2x_2 &\leq 10 \nonumber\\
%	 3x_1 + 8x_2 &\leq 12 \nonumber\\
%	 \text{ where } x_1,x_2 &\geq 0 \nonumber
%	 \end{align}	
%
%\end{enumerate}
%%
%

\caption{}
\label{table:specs}
\end{table}


\section{List of Tasks and Deliverables}
\begin{enumerate}

\item Simulation of suitable Preamble detection algorithm.
\item Simulation of suitable channel estimation and equalization algorithms for mitigating Rayleigh fading channel effects.
\item  Simulation of suitable timing and frequency algorithms for narrow band waveform (coherent TCM-8PSK). Convergence of timing, frequency and equalization algorithms in limited preamble symbols provided (19.5 bytes which is equal to 156 bits or 78 symbols for  TCM 8PSK modulation) and efficient utilization of hardware resources (required all these algorithms shall take max 40\% of FPGA resources).
\item Performance evaluation and Simulation of equalizer and synchronization algorithms should to be done in fixed point.
\item MATLAB codes should be hardware implementable and consider FPGA resources restrictions. Matrix inversions in MATLAB code should not be there.

\end{enumerate}

%Modulator, Demodulator, Encoder and Decoder modules implemented as 
%Software scripts in MATLAB programming environment, that meet the specified performance requirement. Simulation results using the developed software scripts for equalization, timing/frequency synchronization will be presented to support the claims on performance guarantee.

\section{Technology Overview}
Preamble detection can be done using a special kind of correlation as in \cite{frame_offset}.  A symbol detection technique in the presence of time offsets is available in \cite{time_offset}.  A correlation based technique for frequency offset estimation is provided in \cite{freq_offset}. A DFSE based equalization technique based on the Viterbi algorithm is provided in \cite{dfse_viterbi}. A constant modulus decision directed algorithm is proposed in \cite{cmdd}. A channel impulse response based sparse equalization and synchronization technique is given in \cite{cir_sparse}.  Blind equalization techniques are listed in \cite{blind}.

All the above techniques will be analyzed before before finalizing the algorithms for preamble detection, synchronization and equalization.
\bibliography{IEEEabrv,equalizer}
\end{document}
