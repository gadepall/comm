\documentclass[journal,12pt,twocolumn]{IEEEtran}
%
\usepackage{setspace}
\usepackage{gensymb}
\usepackage{xcolor}
\usepackage{caption}
%\usepackage{subcaption}
%\doublespacing
\singlespacing
\usepackage{multicol}

\usepackage{iithtlc}
%\usepackage{graphicx}
%\usepackage{amssymb}
%\usepackage{relsize}
\usepackage[cmex10]{amsmath}
\usepackage{mathtools}
%\usepackage{amsthm}
%\interdisplaylinepenalty=2500
%\savesymbol{iint}
%\usepackage{txfonts}
%\restoresymbol{TXF}{iint}
%\usepackage{wasysym}
\usepackage{amsthm}
\usepackage{mathrsfs}
\usepackage{txfonts}
\usepackage{stfloats}
\usepackage{cite}
\usepackage{cases}
\usepackage{subfig}
%\usepackage{xtab}
\usepackage{longtable}
\usepackage{multirow}
%\usepackage{algorithm}
%\usepackage{algpseudocode}
\usepackage{enumitem}
\usepackage{mathtools}
%\usepackage{stmaryrd}

\usepackage{listings}
    \usepackage[latin1]{inputenc}                                 %%
    \usepackage{color}                                            %%
    \usepackage{array}                                            %%
    \usepackage{longtable}                                        %%
    \usepackage{calc}                                             %%
    \usepackage{multirow}                                         %%
    \usepackage{hhline}                                           %%
    \usepackage{ifthen}                                           %%
  %optionally (for landscape tables embedded in another document): %%
    \usepackage{lscape}     
\usepackage{url}
\def\UrlBreaks{\do\/\do-}

%\usepackage{wasysym}
%\newcounter{MYtempeqncnt}
\DeclareMathOperator*{\Res}{Res}
%\renewcommand{\baselinestretch}{2}
\renewcommand\thesection{\arabic{section}}
\renewcommand\thesubsection{\thesection.\arabic{subsection}}
\renewcommand\thesubsubsection{\thesubsection.\arabic{subsubsection}}

\renewcommand\thesectiondis{\arabic{section}}
\renewcommand\thesubsectiondis{\thesectiondis.\arabic{subsection}}
\renewcommand\thesubsubsectiondis{\thesubsectiondis.\arabic{subsubsection}}

% correct bad hyphenation here
\hyphenation{op-tical net-works semi-conduc-tor}

\def\inputGnumericTable{}  

\lstset{
language=python,
frame=single, 
breaklines=true
}

\begin{document}
%

\theoremstyle{definition}

\newtheorem{theorem}{Theorem}[section]
\newtheorem{problem}{Problem}
\newtheorem{proposition}{Proposition}[section]
\newtheorem{lemma}{Lemma}[section]
\newtheorem{corollary}[theorem]{Corollary}
\newtheorem{example}{Example}[section]
\newtheorem{definition}{Definition}[section]
%\newtheorem{algorithm}{Algorithm}[section]
%\newtheorem{cor}{Corollary}
\newcommand{\BEQA}{\begin{eqnarray}}
\newcommand{\EEQA}{\end{eqnarray}}
\newcommand{\define}{\stackrel{\triangle}{=}}

\bibliographystyle{IEEEtran}
%\bibliographystyle{ieeetr}



\providecommand{\pr}[1]{\ensuremath{\Pr\left(#1\right)}}
\providecommand{\qfunc}[1]{\ensuremath{Q\left(#1\right)}}
\providecommand{\sbrak}[1]{\ensuremath{{}\left[#1\right]}}
\providecommand{\lsbrak}[1]{\ensuremath{{}\left[#1\right.}}
\providecommand{\rsbrak}[1]{\ensuremath{{}\left.#1\right]}}
\providecommand{\brak}[1]{\ensuremath{\left(#1\right)}}
\providecommand{\lbrak}[1]{\ensuremath{\left(#1\right.}}
\providecommand{\rbrak}[1]{\ensuremath{\left.#1\right)}}
\providecommand{\cbrak}[1]{\ensuremath{\left\{#1\right\}}}
\providecommand{\lcbrak}[1]{\ensuremath{\left\{#1\right.}}
\providecommand{\rcbrak}[1]{\ensuremath{\left.#1\right\}}}
\theoremstyle{remark}
\newtheorem{rem}{Remark}
\newcommand{\sgn}{\mathop{\mathrm{sgn}}}
\providecommand{\abs}[1]{\left\vert#1\right\vert}
\providecommand{\res}[1]{\Res\displaylimits_{#1}} 
\providecommand{\norm}[1]{\lVert#1\rVert}
\providecommand{\mtx}[1]{\mathbf{#1}}
\providecommand{\mean}[1]{E\left[ #1 \right]}
\providecommand{\fourier}{\overset{\mathcal{F}}{ \rightleftharpoons}}
%\providecommand{\hilbert}{\overset{\mathcal{H}}{ \rightleftharpoons}}
\providecommand{\system}{\overset{\mathcal{H}}{ \longleftrightarrow}}
\providecommand{\gauss}[2]{\mathcal{N}\ensuremath{\left(#1,#2\right)}}
	%\newcommand{\solution}[2]{\textbf{Solution:}{#1}}
\newcommand{\solution}{\noindent \textbf{Solution: }}
\providecommand{\dec}[2]{\ensuremath{\overset{#1}{\underset{#2}{\gtrless}}}}
%\numberwithin{equation}{section}
%\numberwithin{problem}{section}

\def\putbox#1#2#3{\makebox[0in][l]{\makebox[#1][l]{}\raisebox{\baselineskip}[0in][0in]{\raisebox{#2}[0in][0in]{#3}}}}
     \def\rightbox#1{\makebox[0in][r]{#1}}
     \def\centbox#1{\makebox[0in]{#1}}
     \def\topbox#1{\raisebox{-\baselineskip}[0in][0in]{#1}}
     \def\midbox#1{\raisebox{-0.5\baselineskip}[0in][0in]{#1}}


% paper title
% can use linebreaks \\ within to get better formatting as desired
\title{
\logo{
Interfacing of USRP with Raspberry Pi
}
}
%
%
% author names and IEEE memberships
% note positions of commas and nonbreaking spaces ( ~ ) LaTeX will not break
% a structure at a ~ so this keeps an author's name from being broken across
% two lines.
% use \thanks{} to gain access to the first footnote area
% a separate \thanks must be used for each paragraph as LaTeX2e's \thanks
% was not built to handle multiple paragraphs
%

%\author{Y Aditya, A Rathnakar and G V V Sharma$^{*}$% <-this % stops a space
\author{Prasanna Kumar and G V V Sharma %<-this  stops a space
\thanks{The authors are with the Department
of Electrical Engineering, IIT, Hyderabad
502285 India e-mail: \{gadepall\}@iith.ac.in. 
}}



% make the title area
\maketitle

\tableofcontents

\bigskip

\begin{abstract}
This manual shows how to  interface USRP B210 with Raspberry pi. 
\end{abstract}

\section{Installation of UHD from Source}
In this section the installation of UHD from source on Ubuntu MATE and Raspbian OS explained as following steps.
\subsection{Install Dependences}
Follow installation instructions for Ubuntu 14.04 in \cite{ettus}.  Copy paste the instructions on the linux terminal and press enter.
%\begin{itemize}
%\item sudo apt-get update
%\item \textbf{On Raspbian -- }\\
%sudo apt-get -y install git swig cmake doxygen build-essential libboost-all-dev libtool libusb-1.0-0 libusb-1.0-0-dev libudev-dev libncurses5-dev libfftw3-bin libfftw3-dev libfftw3-doc libcppunit-1.13-0 libcppunit-dev libcppunit-doc ncurses-bin cpufrequtils python-numpy python-numpy-doc python-numpy-dbg python-scipy python-docutils qt4-bin-dbg qt4-default qt4-doc libqt4-dev libqt4-dev-bin python-qt4 python-qt4-dbg python-qt4-dev python-qt4-doc python-qt4-doc libfftw3-bin libfftw3-dev libfftw3-doc ncurses-bin libncurses5 libncurses5-dev libncurses5-dbg   libfontconfig1-dev libxrender-dev libpulse-dev swig g++ automake autoconf libtool python-dev libfftw3-dev libcppunit-dev libboost-all-dev libusb-dev libusb-1.0-0-dev fort77 libsdl1.2-dev python-wxgtk2.8 git-core libqt4-dev python-numpy ccache python-opengl libgsl0-dev python-cheetah python-mako python-lxml doxygen qt4-default qt4-dev-tools libusb-1.0-0-dev libqwt5-qt4-dev libqwtplot3d-qt4-dev pyqt4-dev-tools python-qwt5-qt4 cmake git-core wget libxi-dev gtk2-engines-pixbuf r-base-dev python-tk liborc-0.4-0 liborc-0.4-dev libasound2-dev python-gtk2 libzmq1 libzmq-dev python-requests python-sphinx libcomedi-dev
%
%\item \textbf{On Ubuntu MATE-- }\\
%sudo apt-get -y install git swig cmake doxygen build-essential libboost-all-dev libtool libusb-1.0-0 libusb-1.0-0-dev libudev-dev libncurses5-dev libfftw3-bin libfftw3-dev libfftw3-doc libcppunit-1.13-0v5 libcppunit-dev libcppunit-doc ncurses-bin cpufrequtils python-numpy python-numpy-doc python-numpy-dbg python-scipy python-docutils qt4-bin-dbg qt4-default qt4-doc libqt4-dev libqt4-dev-bin python-qt4 python-qt4-dbg python-qt4-dev python-qt4-doc python-qt4-doc libqwt6abi1 libfftw3-bin libfftw3-dev libfftw3-doc ncurses-bin libncurses5 libncurses5-dev libncurses5-dbg libfontconfig1-dev libxrender-dev libpulse-dev swig g++ automake autoconf libtool python-dev libfftw3-dev libcppunit-dev libboost-all-dev libusb-dev libusb-1.0-0-dev fort77 libsdl1.2-dev python-wxgtk3.0 git-core libqt4-dev python-numpy ccache python-opengl libgsl-dev python-cheetah python-mako python-lxml doxygen qt4-default qt4-dev-tools libusb-1.0-0-dev libqwt5-qt4-dev pyqt4-dev-tools python-qwt5-qt4 cmake git-core wget libxi-dev gtk2-engines-pixbuf r-base-dev python-tk liborc-0.4-0 liborc-0.4-dev libasound2-dev python-gtk2 libzmq-dev libzmq1 python-requests python-sphinx libcomedi-dev python-zmq
%\end{itemize}

\subsection{Building and installing UHD from source code}
\begin{itemize}
%
\item Update the system condition.
\begin{lstlisting}
sudo apt-get update
\end{lstlisting}
%
\item Create a directory in the HOME directory of Pi
\begin{lstlisting}[frame=single]
mkdir workarea-uhd
cd workarea-uhd
\end{lstlisting}
%
\item Cloning of uhd files from github 
\begin{lstlisting}[frame=single]
git clone https://github.com/EttusResearch/uhd
cd uhd
\end{lstlisting}
%
\item Listing out the various releases and checking the release required is present or not. Here meed 3.10.1 release.
\begin{lstlisting}[frame=single]
git tag -l
git checkout release_003_010_001_000
\end{lstlisting}
%\pagebreak
\item Make another directory in host.
\begin{lstlisting}[frame=single]
cd host
mkdir build
cd build
\end{lstlisting}
%
\item Creating Cmake files 
\begin{lstlisting}[frame=single]
cmake ../
\end{lstlisting}
%
\item Building of UHD
\begin{lstlisting}[frame=single]
make 
\end{lstlisting}
%
\item conducting some basic tests to verify that the build process completed properly. 
\begin{lstlisting}[frame=single]
make test
\end{lstlisting}
%
\item Installing UHD
\begin{lstlisting}[frame=single]
sudo make install
\end{lstlisting}
%
\item  update the system's shared library cache.
\begin{lstlisting}[frame=single]
sudo ldconfig
\end{lstlisting}
%
\item Loading the UHD images, This task should performed by going to HOME directory of Pi 
\begin{lstlisting}[frame=single]
sudo uhd_images_downloader
\end{lstlisting}
%
\item Setting of path for images, Sometime this has to be done when ever you reconnect the USRP.
\begin{lstlisting}[frame=single]
export LD_LIBRARY_PATH=$LD_LIBRARY_PATH:/usr/local/lib
\end{lstlisting}
%
\item Command to find the details of uhd devices connected to Raspberry pi
\begin{lstlisting}[frame=single]
sudo uhd_find_devices
\end{lstlisting}
%
\end{itemize}
%
%\pagebreak
%\section{Basic Programing of UHD and C++ }
%In this section C++ programs are created to transmit or receive signals from USRP. This programs are bassed on different communication techniques 

%\begin{thebibliography}{00}
%\bibitem{b1} Ettus Research, "Building and Installing the USRP Open-Source Toolchain (UHD and GNU Radio) on Linux," $https://kb.ettus.com/Building_and_Installing_the_USRP_Open-Source_Toolchain_(UHD_and_GNU_Radio)_on_Linux$.
%\end{thebibliography}

\bibliography{IEEEabrv,usrp}
\end{document}